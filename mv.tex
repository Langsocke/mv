\documentclass[a4paper,12pt,titlepage]{scrartcl}

%Pakete
%\usepackage[left=3cm,right=2cm,top=2cm,bottom=3cm]{geometry} %Seitenränder
\usepackage[utf8]{inputenc} % ermöglicht die direkte Eingabe der Umlaute
\usepackage [german]{babel} %Spracheinstellungen

\usepackage {csquotes} % Anführungszeichen nach dem Stil \enquote{ich bin zitiert.}
\usepackage{subscript} % erlaubt Tiefstellen von Zahlen und Text
\usepackage{tabularx} %krassere tabellen
\usepackage{datetime}
\usepackage{needspace}
\usepackage{tikz}
\sloppy %macht ungefähren Blocksatz, wenn nichts anderes an Trennhilfen was nützt

% Hurenkinder und Schusterjungen verhindern
\clubpenalty10000
\widowpenalty10000
\displaywidowpenalty=10000

\def \logo {
	\begin{picture}(0,0)
		\put(-9, 2){\parbox{180mm}{
				%% Inline Logo
				\begin{tikzpicture}[
						y=0.8pt,
						x=0.8pt,
						yscale=-1,
						inner sep=0pt,
						outer sep=0pt,
						scale=0.2
					]
					\begin{scope}[shift={(-448.05262,-67.70133)}]
						\path[color=black,fill=black,even odd rule,line width=16.000pt]
						(534.1205,161.8816) -- (534.1205,172.2877) -- (581.3732,172.2877) .. controls
						(571.8452,181.8308) and (562.3036,191.3603) .. (552.7463,200.8741) --
						(565.3389,200.8741) .. controls (585.4511,200.8741) and (601.7400,217.2036) ..
						(601.7400,237.3158) .. controls (601.7400,257.4280) and (585.4511,273.7170) ..
						(565.3389,273.7170) .. controls (545.2267,273.7170) and (528.8972,257.4280) ..
						(528.8972,237.3158) -- (518.4911,237.3158) .. controls (518.4911,263.1743) and
						(539.4804,284.1231) .. (565.3389,284.1231) .. controls (591.1974,284.1231) and
						(612.1462,263.1743) .. (612.1462,237.3158) .. controls (612.1462,215.3233) and
						(596.9939,196.8355) .. (576.5548,191.8042) -- (606.5180,161.8816) -- cycle;
						\begin{scope}[cm={{-1.0,0.0,0.0,-1.0,(1130.6373,475.18252)}}]
							\path[color=black,fill=black,even odd rule,line width=16.000pt]
							(565.3389,208.6889) .. controls (549.5365,208.6889) and (536.7120,221.5134) ..
							(536.7120,237.3158) .. controls (536.7120,249.4467) and (544.2722,259.8031) ..
							(554.9328,263.9587) -- (554.9328,252.2164) .. controls (550.2090,248.9346) and
							(547.1181,243.5056) .. (547.1181,237.3158) .. controls (547.1181,227.2597) and
							(555.2828,219.0950) .. (565.3389,219.0950) .. controls (575.3950,219.0950) and
							(583.5192,227.2597) .. (583.5192,237.3158) .. controls (583.5192,243.4955) and
							(580.4441,248.9328) .. (575.7450,252.2164) -- (575.7450,263.9587) .. controls
							(586.3945,259.8031) and (593.9253,249.4467) .. (593.9253,237.3158) .. controls
							(593.9253,221.5134) and (581.1413,208.6889) .. (565.3389,208.6889) -- cycle;
							\path[color=black,fill=black,even odd rule,line width=16.000pt]
							(565.3389,232.0925) .. controls (562.4657,232.0925) and (560.1156,234.4426) ..
							(560.1156,237.3158) -- (560.1156,268.4937) -- (570.5217,268.4937) --
							(570.5217,237.3158) .. controls (570.5217,234.4426) and (568.2121,232.0925) ..
							(565.3389,232.0925) -- cycle;
						\end{scope}
						\path[fill=black] (630.8529,149.4914) .. controls (625.1306,149.5458) and
						(620.4890,150.3447) .. (619.3535,151.0301) .. controls (617.0825,152.4009) and
						(616.5461,155.6877) .. (617.2480,156.4558) .. controls (617.9500,157.2237) and
						(628.6785,164.1143) .. (633.3633,166.3761) .. controls (645.9778,172.4665) and
						(632.0756,195.7720) .. (621.2161,191.6018) .. controls (613.5199,188.6464) and
						(604.7892,179.8446) .. (603.3192,179.7785) .. controls (601.8492,179.7137) and
						(599.9557,180.7048) .. (599.4726,183.2607) .. controls (598.9895,185.8166) and
						(602.4810,198.1576) .. (611.2554,207.7576) .. controls (621.7846,219.2778) and
						(638.1817,206.5491) .. (638.1817,217.0704) --
						(638.1817,265.9427)arc(179.961:0.039:13.018) -- (664.2172,203.5060) --
						(664.2172,188.9294) .. controls (664.2172,183.4623) and (670.9201,161.5406) ..
						(648.7498,152.3663) .. controls (643.3754,150.1424) and (636.5752,149.4370) ..
						(630.8529,149.4914) -- cycle;
					\end{scope}
					\path[color=black,fill=black,even odd rule,line width=8.333pt]
					(150.0000,23.1213) .. controls (79.9291,23.1213) and (23.1213,79.9291) ..
					(23.1213,150.0000) .. controls (23.1213,220.0708) and (79.9291,276.8787) ..
					(150.0000,276.8787) .. controls (220.0708,276.8787) and (276.8787,220.0708) ..
					(276.8787,150.0000) .. controls (276.8787,79.9291) and (220.0708,23.1213) ..
					(150.0000,23.1213) -- cycle(150.0000,40.5068) .. controls (210.4603,40.5068)
					and (259.4540,89.5397) .. (259.4540,150.0000) .. controls (259.4540,210.4603)
					and (210.4603,259.4540) .. (150.0000,259.4540) .. controls (89.5278,259.8428)
					and (41.3547,203.5210) .. (40.5068,150.0000) .. controls (40.5068,89.5397) and
					(89.5397,40.5068) .. (150.0001,40.5068) -- cycle;
				\end{tikzpicture}
				%% Logo End
		}}
	\end{picture}
}

%\usepackage[final]{graphicx}
\usepackage{floatflt}

\addto\captionsgerman{%
  \renewcommand{\contentsname}%
    {Tagesordnung}%
}

\def \thetitle {Mitgliederversammlung des Netz39 \nolinebreak e.V.}

\title{ \logo \\ \vspace{0.2\baselineskip} \thetitle}
\author{
Leitung: Stefan Haun \\
Protokollant/Schriftführer: David Kilias \\
Ort der Versammlung:\\ Netz39 e.V., Leibnizstraße 32, 39104 Magdeburg \\
}
\date{\displaydate{date}} % sollte man u.U. anpassen ;)

\usepackage{fancyhdr}
\setlength{\headheight}{3\baselineskip}
\pagestyle{fancy}
\renewcommand{\headrulewidth}{0pt}

\fancyhead[EL,OL]{
	\logo
}
\fancyhead[EC,OC] {
	\thetitle}
\fancyhead[ER,OR] {
	\displaydate{date}
}
\newdate{date}{15}{10}{2014}

\newcommand{\todo}[1]{\marginpar{#1}}
\usepackage[colorlinks=true,linkcolor=black,citecolor=black,urlcolor=black,breaklinks=true]{hyperref}



\begin{document}
\maketitle
\tableofcontents

\clearpage
\section*{Begrüßung}
Der Vorsitzende des Vereins (Stefan Haun) begrüßt die zur Versammlung anwesenden Mitglieder in den Räumen des Netz39 e.V.
Die Vereinsmitglieder werden nochmals über den Grund der ausserplanmäßigen Mitgliederversammlung aufgeklärt: Bei der letzten Mitgliederversammlung gab es Formfehler bei der Wahl des Vorstands und der Kassenprüfer, durch die die Wahl erneut durchgeführt werden muss.

\section{Feststellung der Beschlussfähigkeit}
Der Verein hat zum Zeitpunkt der Mitgliederversammlung 50 Mitglieder. Es sind 29 von 48 stimmberechtigten Mitgliedern des Vereins und ein Gast anwesend. Für eine einfache Mehrheit sind 15 Stimmen nötig. Die Tagesordnung (siehe Anhang) wurde mit der Einladung versandt. Es wurde Satzungsgemäß eingeladen. Die Beschlussfähigkeit wird festgestellt.

\section{Bericht über die Tätigkeit des Vorstands}
Der Vorstand berichtet über Formalien und Tagesgeschäft der letzten Amtszeit.

\section{Entlastung des Vorstands}
Die Entlastung des Vorstands (Stefan Haun) wird mit einer Enthaltung angenommen.
Die Entlastung des Schatzmeisters (David Kilias) wird vorbehaltlich der Kassenprüfung einstimmig angenommen.

\section{Wahlen}
\subsection{Wahl des Vorstands}
Die Wahlen des Vorstandes wurden durch eine Wahl zur Bestimmung eines geeigneten Kanditaten begonnen, der dann durch Abstimmung bestätigt wurde.

\begin{center}
	\begin{tabular}{l | r | r}
			Kandidat & Stimmen Vorstand & Stimmen Stellvertreter \\ \hline
			Sebatian Mai & 3 & 18 \\
			René Meye & 17 & - \\
			Tatjana Böttger & 4 & 7 \\
			Stefan Haun & 2 & - \\
		\end{tabular}
\end{center}

\subsubsection*{Ergebnis der Abstimmung}
René Meye mit einer Gegenstimme und drei Enthaltungen als Vorsitzender gewählt. \\
Sebastian Mai mit einer Gegenstimme und sechs Enthaltungen als Stellvertretender Vorsitzender gewählt.
Beide nehmen die Wahl an.

\subsection{Wahl des Schatzmeisters}
Kandidaten:
\begin{itemize}
	\item David Kilias
\end{itemize}
David Kilias mit einer Enthaltung einstimmig als Schatzmeister gewählt und nimmt die Wahl an.

\subsection{Wahl der Kassenprüfer}
Kandidaten:
\begin{itemize}
  \item Katharina Holstein
  \item Alexander Dahl
\end{itemize}
Katharina Holstein wird einstimmig zum Kassenprüfer gewählt. \\
Alexander Dahl mit drei Enthaltungen einstimmig zum Kassenprüfer gewählt. Beide nehmen die Wahl an.

\section{Beschlussfassung über ein Standardformular zur Aufnahme natürlicher Personen als Fördermitglieder zur Erfüllung von § 1
Abs. 5 der Beitrags- und Mahnordnung}
Die Mitgliederversammlung beauftragt den Vorstand ein Formular zu erstellen,	welches die Aufnahme von natürlichen Fördermitglieder satzungsgemäß regelt.
Die Mitgliederversammlung beschließt die Aufnahme von Fördermitgliedern,	welche durch dieses Formular für die Aufnahme von natürlichen Mitgliedern als Fördermitgliedern beantragt wurden.
Der Antrag wird mit einer Enthaltung einstimmig angenommen.

\section{Beschlussfassung über ggf. vorliegende Anträge}
Keine Anträge liegen vor, damit wird die Mitgliederversammlung geschlossen.

\nopagebreak
\vspace{10\baselineskip}
\begin{tabularx}{\textwidth}[b]{X X}
	\hline
	David Kilias & René Meye \\
	Schatzmeister und Protokollant & Vorsitzender
\end{tabularx}

\appendix
Stub.


\end{document}
