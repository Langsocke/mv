\documentclass[a4paper,12pt,titlepage]{scrartcl}

%Pakete
%\usepackage[left=3cm,right=2cm,top=2cm,bottom=3cm]{geometry} %Seitenränder
\usepackage[utf8]{inputenc} % ermöglicht die direkte Eingabe der Umlaute
\usepackage [german]{babel} %Spracheinstellungen

\usepackage {csquotes} % Anführungszeichen nach dem Stil \enquote{ich bin zitiert.}
\usepackage{subscript} % erlaubt Tiefstellen von Zahlen und Text
\usepackage{tabularx} %krassere tabellen
\usepackage{datetime}
\usepackage{needspace}
\usepackage{tikz}
\usepackage{eurosym}
\sloppy %macht ungefähren Blocksatz, wenn nichts anderes an Trennhilfen was nützt

% Hurenkinder und Schusterjungen verhindern
\clubpenalty10000
\widowpenalty10000
\displaywidowpenalty=10000

\def \logo {
	\begin{picture}(0,0)
		\put(-9, 2){\parbox{180mm}{
				%% Inline Logo
				\begin{tikzpicture}[
						y=0.8pt,
						x=0.8pt,
						yscale=-1,
						inner sep=0pt,
						outer sep=0pt,
						scale=0.2
					]
					\begin{scope}[shift={(-448.05262,-67.70133)}]
						\path[color=black,fill=black,even odd rule,line width=16.000pt]
						(534.1205,161.8816) -- (534.1205,172.2877) -- (581.3732,172.2877) .. controls
						(571.8452,181.8308) and (562.3036,191.3603) .. (552.7463,200.8741) --
						(565.3389,200.8741) .. controls (585.4511,200.8741) and (601.7400,217.2036) ..
						(601.7400,237.3158) .. controls (601.7400,257.4280) and (585.4511,273.7170) ..
						(565.3389,273.7170) .. controls (545.2267,273.7170) and (528.8972,257.4280) ..
						(528.8972,237.3158) -- (518.4911,237.3158) .. controls (518.4911,263.1743) and
						(539.4804,284.1231) .. (565.3389,284.1231) .. controls (591.1974,284.1231) and
						(612.1462,263.1743) .. (612.1462,237.3158) .. controls (612.1462,215.3233) and
						(596.9939,196.8355) .. (576.5548,191.8042) -- (606.5180,161.8816) -- cycle;
						\begin{scope}[cm={{-1.0,0.0,0.0,-1.0,(1130.6373,475.18252)}}]
							\path[color=black,fill=black,even odd rule,line width=16.000pt]
							(565.3389,208.6889) .. controls (549.5365,208.6889) and (536.7120,221.5134) ..
							(536.7120,237.3158) .. controls (536.7120,249.4467) and (544.2722,259.8031) ..
							(554.9328,263.9587) -- (554.9328,252.2164) .. controls (550.2090,248.9346) and
							(547.1181,243.5056) .. (547.1181,237.3158) .. controls (547.1181,227.2597) and
							(555.2828,219.0950) .. (565.3389,219.0950) .. controls (575.3950,219.0950) and
							(583.5192,227.2597) .. (583.5192,237.3158) .. controls (583.5192,243.4955) and
							(580.4441,248.9328) .. (575.7450,252.2164) -- (575.7450,263.9587) .. controls
							(586.3945,259.8031) and (593.9253,249.4467) .. (593.9253,237.3158) .. controls
							(593.9253,221.5134) and (581.1413,208.6889) .. (565.3389,208.6889) -- cycle;
							\path[color=black,fill=black,even odd rule,line width=16.000pt]
							(565.3389,232.0925) .. controls (562.4657,232.0925) and (560.1156,234.4426) ..
							(560.1156,237.3158) -- (560.1156,268.4937) -- (570.5217,268.4937) --
							(570.5217,237.3158) .. controls (570.5217,234.4426) and (568.2121,232.0925) ..
							(565.3389,232.0925) -- cycle;
						\end{scope}
						\path[fill=black] (630.8529,149.4914) .. controls (625.1306,149.5458) and
						(620.4890,150.3447) .. (619.3535,151.0301) .. controls (617.0825,152.4009) and
						(616.5461,155.6877) .. (617.2480,156.4558) .. controls (617.9500,157.2237) and
						(628.6785,164.1143) .. (633.3633,166.3761) .. controls (645.9778,172.4665) and
						(632.0756,195.7720) .. (621.2161,191.6018) .. controls (613.5199,188.6464) and
						(604.7892,179.8446) .. (603.3192,179.7785) .. controls (601.8492,179.7137) and
						(599.9557,180.7048) .. (599.4726,183.2607) .. controls (598.9895,185.8166) and
						(602.4810,198.1576) .. (611.2554,207.7576) .. controls (621.7846,219.2778) and
						(638.1817,206.5491) .. (638.1817,217.0704) --
						(638.1817,265.9427)arc(179.961:0.039:13.018) -- (664.2172,203.5060) --
						(664.2172,188.9294) .. controls (664.2172,183.4623) and (670.9201,161.5406) ..
						(648.7498,152.3663) .. controls (643.3754,150.1424) and (636.5752,149.4370) ..
						(630.8529,149.4914) -- cycle;
					\end{scope}
					\path[color=black,fill=black,even odd rule,line width=8.333pt]
					(150.0000,23.1213) .. controls (79.9291,23.1213) and (23.1213,79.9291) ..
					(23.1213,150.0000) .. controls (23.1213,220.0708) and (79.9291,276.8787) ..
					(150.0000,276.8787) .. controls (220.0708,276.8787) and (276.8787,220.0708) ..
					(276.8787,150.0000) .. controls (276.8787,79.9291) and (220.0708,23.1213) ..
					(150.0000,23.1213) -- cycle(150.0000,40.5068) .. controls (210.4603,40.5068)
					and (259.4540,89.5397) .. (259.4540,150.0000) .. controls (259.4540,210.4603)
					and (210.4603,259.4540) .. (150.0000,259.4540) .. controls (89.5278,259.8428)
					and (41.3547,203.5210) .. (40.5068,150.0000) .. controls (40.5068,89.5397) and
					(89.5397,40.5068) .. (150.0001,40.5068) -- cycle;
				\end{tikzpicture}
				%% Logo End
		}}
	\end{picture}
}

%\usepackage[final]{graphicx}
\usepackage{floatflt}

\addto\captionsgerman{%
  \renewcommand{\contentsname}%
    {Tagesordnung}%
}

\def \thetitle {Mitgliederversammlung des Netz39 \nolinebreak e.\,V.}

\title{ \logo \\ \vspace{0.2\baselineskip} \thetitle}
\author{
Leitung: René Meye\\
Protokollant/Schriftführer: Sebastian Mai \\
Ort der Versammlung:\\ Netz39 e.V., Leibnizstraße 32, 39104 Magdeburg \\
}
\date{\displaydate{date}} % sollte man u.U. anpassen ;)

\usepackage{fancyhdr}
\setlength{\headheight}{3\baselineskip}
\pagestyle{fancy}
\renewcommand{\headrulewidth}{0pt}

\fancyhead[EL,OL]{
	\logo
}
\fancyhead[EC,OC] {
	\thetitle}
\fancyhead[ER,OR] {
	\displaydate{date}
}
\newdate{date}{14}{10}{2015}

\usepackage[colorlinks=true,linkcolor=black,citecolor=black,urlcolor=black,breaklinks=true]{hyperref}
\usepackage[colorinlistoftodos]{todonotes}

\begin{document}
\maketitle
\tableofcontents

\clearpage

\section{Begrüßung und Feststellung der Beschlussfähigkeit}
Der Vorsitzende des Vereins (René Meye) begrüßt die zur Versammlung anwesenden Mitglieder in den Räumen des Netz39 e.\,V.

\subsection{Feststellung der Beschlussfähigkeit}
Der Verein hat zum Zeitpunkt der Mitgliederversammlung 49 Mitglieder. Es sind 18 von 47 stimmberechtigten Mitgliedern des Vereins und ein Gast anwesend. Die Tagesordnung (siehe Anhang) wurde mit der Einladung versandt. Es wurde satzungsgemäß eingeladen. Die Beschlussfähigkeit wird festgestellt.

\subsection{Beschluss über Änderungsanträge zur Tagesordnung}
Update zum Änderungsantrag in TOP 5.

\section{Bericht über die Tätigkeit des Vorstands}
Der Vorstand berichtet über Formalien und Tagesgeschäft der letzten Amtszeit von Oktober 2014 bis zum Zeitpunkt der Mitgliederversammlung.
Am Anfang der Vorstandsarbeit stand eine Beschäftigung mit der Zufriedenheit im Vereinsleben: Gespräche zur Vermittlung zwischen Mitgliedern wurden geführt und das Vereinsklima hat sich gebessert.
Die Finanzierung des Vereins ist ausreichend für die laufenden Kosten. Der Vorstand führt eine monatliche Projektförderung ein, durch die Projekte im Sinne des Vereins unterstützt werden.

Des weiteren wurde ein neuer Server für den Verein angemietet, um diese wichtige Infrastruktur in die Hände des Vereins zu legen. Danke an Andreas Pfohl, der sich um die Administration des Servers kümmert.
Im Frühjahr hat sich der Verein nicht unwesentlich an der Durchführung und Planung der Easterhegg 2015 beteiligt. Außerdem wurde der Bestform Wettbewerb durch Sebastian Hichert gewonnen, Netz39 trat als Industriepartner auf.
Das Ende der Vorstandslegislatur wurde erschwert durch die Arbeitsumstände der Vorstandsmitglieder, der Vorstand entschuldigt sich dafür beim Verein.

Für die Zukunft ist der Abschluss einer Vereinshaftpflichtversicherung beschlossen, der Abschluss dieser Versicherung wird durch den neuen Vorstand erfolgen.
Der Vorstand und der Verein begrüßen, dass sowohl UCD+ als auch Marmalade den Verein als Fördermitglieder unterstützen.
Zurückgegangen ist der Betrag der regelmäßig zugesicherten Spenden. Dieser Rückgang ist durch die bessere finanzielle Situation des Vereins erklärbar.

Die Mitgliedsbeiträge reichen ungefähr zur Deckung der laufenden Kosten aus.
Leider ist der Händler, bei dem der Verein einen Beamer kaufen wollte nicht erreichbar, ca. \EUR{535} Rückbuchung sind offen und das Erstatten einer Anzeige steht noch aus.
Des weiteren steht auch die finanzielle Abwicklung der Easterhegg noch aus.

\section{Entlastung des Vorstands}

\subsection{Bericht der Kassenprüfer}
Es haben zwei Kassenprüfungen stattgefunden, zum 31.12.2014 und zum 10.10.2015.
Ergebnis der ersten Kassenprüfung: Es gibt nun keine offenen Posten bei privaten Kreditoren mehr. 
Die zweite Kassenprüfung ergab, dass die finanzielle Abwicklung der Easterhegg noch aussteht. Außerdem wurden versehentlich 5\,ct zu viel überwiesen, dieser Fehlbetrag wird beglichen.
Die Kassenprüfer Katharina Holstein und Alexander Dahl empfehlen eine Entlastung des Schatzmeisters.

\subsection{Entlastung des Vorstandes}
\begin{tabularx}{\textwidth}[b]{l | l | X | X | X}
	Vorstandsmitglied & Posten & 	Dafür & Dagegen & Enthaltungen \\
	\hline
	René Meye & Vorstandsvorsitzender & 18 & 0 & 0 \\
	Sebastian Mai & Stellvertretender Vorsitzender & 17 & 0 & 1 \\
	David Kilias & Schatzmeister & 18 & 0 & 0 \\
\end{tabularx} \\ \\
Die Mitgliederversammlung beschließt, den bisherigen Vorstand zu entlasten. Eine Kassenprüfung fand statt.

Jens Winter verlässt die Versammlung.

\section{Wahlen}

Zunächst wird erklärt, welche Aufgaben der neue Vorstand erfüllen soll und wie die Wahl abläuft. Wahlleiter ist Stefan Haun (einstimmig beschlossen).

\subsection{Wahl des Vorstandsvorsitzenden und stellvertretenden Vorsitzenden}
Der Vorstandsvorsitzende wird auf Antrag eines Mitgliedes in geheimer Wahl gewählt.
Nominiert für Vorsitzenden sind Michel Vorsprach und Sebastian Mai.
Die Auszählung ergab: 10 Stimmen für Michel Vorsprach, 7 Stimmen für Sebastian Mai, 0 Enthaltungen. Michel Vorsprach nimmt die Wahl zum Vorstandsvorsitzenden des Netz39 an.

Nominiert für den Posten des stellvertretenden Vorsitzenden ist Sebastian Mai.
Mit 16 Stimmen dafür und einer Enthaltung wird Sebastian Mai zum Stellvertretenden Vorsitzenden Gewählt.
Er nimmt die Wahl an.

\subsection{Wahl des Schatzmeisters}
Nominiert für den Posten des Schatzmeisters ist Carina Lederich. Sie wird mit 16 Stimmen dafür und einer Enthaltung gewählt.
Carina Lederich nimmt die Wahl an.

\subsection{Wahl der Kassenprüfer}
Nominiert sind Nadine Kempe, Katharina Lehmann und Uwe Hermsdorf. \\
\begin{tabularx}{\textwidth}[b]{l | X | X }
	Name & 1. Wahlgang & 2. Wahlgang \\ \hline
	Nadine Kempe & 0 & 5 \\
	Katharina Lehmann & 5 & 12 \\
	Uwe Hermsdorf & 12 & – \\
	Enthaltungen & 0 & 0
\end{tabularx} \\ \\

Im ersten Wahlgang wird Uwe Hermsdorf zum Kassenprüfer gewählt, im zweiten Wahlgang wird Katharina Lehmann gewählt.
Beide nehmen die Wahl an.

\section{Beschlussfassung über einen Änderungsantrag zur \enquote{Ermöglichung eines sofortigen Vereinsaustritts} nach https://github.com/netz39/Ordnungen/pull/1/}
\subsection{Vorstellung es Satzungänderungsantrag 1.1}

Änderung von §1 Absatz 3 der Beitrags und Mahnordnung in:
\begin{displayquote}
3. Der Quartalsbeitrag beträgt für aktive Mitglieder \EUR{90,00}.
\end{displayquote}
§1 Absatz 4 der Beitrags und Mahnordnung in:
\begin{displayquote}
4. Auf Wunsch eines Mitglieds wird ein ermäßigter Beitragssatz in Höhe von \EUR{30,00} pro Quartal gewährt. Dieser Wunsch kann rechtzeitig vor Beginn eines Quartals beim Schatzmeister für ein Kalenderjahr angezeigt werden. Alle Mitglieder sind angehalten, diese Regelung nur im Bedarfsfall zu nutzen.
\end{displayquote}
Hinzufügen des Absatzes 6 zur Beitrags und Mahnordnung §1:
\begin{displayquote}
6. Der Mitgliedsbeitrag im Eintrittsquartal entfällt. Alle Neumitglieder sind angehalten, einen für sie angemessenen Betrag zu spenden.
\end{displayquote}
Änderung von §4 Absatz 3 Punkt 1 der Satzung in:
\begin{displayquote}
3. Die Mitgliedschaft endet – durch freiwillige Beendigung ohne Frist durch schriftliche Erklärung gegenüber dem Vorstand;
\end{displayquote}

Abstimmung über Satzungänderungsantrags 1.1: 17 Stimmen dagegen.

\subsection{Vorstellung es Satzungänderungsantrag 2}

Änderung von §4 Absatz 3 Punkt 1 der Satzung in:
\begin{displayquote}
3. Die Mitgliedschaft endet – durch freiwillige Beendigung mit zweiwöchiger Frist zum Ende des Monats durch schriftliche Erklärung gegenüber dem Vorstand;
\end{displayquote}
Abstimmung über Satzungsänderungsantrag 2: 16 Stimmen dagegen, 1 Enthaltung.

\subsection{Vorstellung es Satzungänderungsantrag 3}
Änderung von §4 Absatz 3 Punkt 1 der Satzung in:

\begin{displayquote}
3. Die Mitgliedschaft endet 
	- durch freiwillige Beendigung ohne Frist durch schriftliche Erklärung gegenüber dem Vorstand;
\end{displayquote}

Sowie des §1 der Beitragsordnung in: 
\begin{displayquote}
1. Der Mitgliedsbeitrag ist jeweils zum Beginn eines Kalenderquartals für das gesamte Quartal im Voraus zu entrichten oder kann per Lastschriftverfahren eingezogen werden.

2. Im Falle eines Beitritts innerhalb eines Quartals, wird eine Ermäßigung des Quartalsbeitrages gewährt. Diese beträgt für jeden bereits abgelaufenen Kalendermonat des aktuellen Quartals ein Drittel des Quartalsbeitrages.

3. Für den Beitritt wird eine Beitrittsgebühr in Höhe von \EUR{10,00} erhoben, die mit dem ersten Beitrag fällig wird.

4. Der Quartalsbeitrag beträgt für aktive Mitglieder \EUR{90,00}.

5. Auf Wunsch eines Mitglieds wird ein ermäßigter Beitragssatz in Höhe von \EUR{30,00} pro Quartal gewährt. Dieser Wunsch kann rechtzeitig vor Beginn eines Quartals beim Schatzmeister für ein Kalenderjahr angezeigt werden. Alle Mitglieder sind angehalten, diese Regelung nur im Bedarfsfall zu nutzen.

6. Der Beitrag für Fördermitglieder wird individuell vom Vorstand mit den jeweiligen Fördermitgliedern ausgehandelt und auf der nächsten Mitgliederversammlung bestätigt. Bei natürlichen Personen dienen Abs. 3 und Abs. 4 als Richtlinie.
\end{displayquote}

Abstimmung über Satzungsänderungsantrag 3: 17 Stimmen dagegen.

\subsection{Vorstellung des Satzungänderungsantrag 4}
Änderung von §4  Absatz 3 Punkt 1 der Satzung in:
\begin{displayquote}
3. Die Mitgliedschaft endet – durch freiwillige Beendigung mit zweiwöchiger Frist zum Ende des Quartals durch schriftliche Erklärung gegenüber dem Vorstand;
\end{displayquote}

Abstimmung über Satzungsänderungsantrag 4: 15 Stimmen dafür, 2 Enthaltungen. Damit ist der Satzungsänderungsantrag 4 angenommen.

\section{Beschlussfassung über ggf. vorliegende Anträge}

Der Antrag wird vorgestellt: \enquote{Das Netz39 möge beschließen das Projekt \enquote{Eigenbau einer eigenen CNC-Fräse} zu unterstützen.}.

Andreas Pfohl verlässt die Mitgliederversammlung (22:46 Uhr).

Der Beschluss wurde abgelehnt mit 6 Stimmen dafür, 4 Enthaltungen und 6 Stimmen dagegen.

\nopagebreak
\vspace{10\baselineskip}
\begin{tabularx}{\textwidth}[b]{X X}
	\hline
	Sebastian Mai & Michel Vorsprach \\
	Protokollant & Vorsitzender
\end{tabularx}

\appendix
Stub.


\end{document}
